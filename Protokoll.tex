% Options for packages loaded elsewhere
\PassOptionsToPackage{unicode}{hyperref}
\PassOptionsToPackage{hyphens}{url}
\PassOptionsToPackage{dvipsnames,svgnames,x11names}{xcolor}
%
\documentclass[
  letterpaper,
  DIV=11,
  numbers=noendperiod]{scrartcl}

\usepackage{amsmath,amssymb}
\usepackage{iftex}
\ifPDFTeX
  \usepackage[T1]{fontenc}
  \usepackage[utf8]{inputenc}
  \usepackage{textcomp} % provide euro and other symbols
\else % if luatex or xetex
  \usepackage{unicode-math}
  \defaultfontfeatures{Scale=MatchLowercase}
  \defaultfontfeatures[\rmfamily]{Ligatures=TeX,Scale=1}
\fi
\usepackage{lmodern}
\ifPDFTeX\else  
    % xetex/luatex font selection
\fi
% Use upquote if available, for straight quotes in verbatim environments
\IfFileExists{upquote.sty}{\usepackage{upquote}}{}
\IfFileExists{microtype.sty}{% use microtype if available
  \usepackage[]{microtype}
  \UseMicrotypeSet[protrusion]{basicmath} % disable protrusion for tt fonts
}{}
\makeatletter
\@ifundefined{KOMAClassName}{% if non-KOMA class
  \IfFileExists{parskip.sty}{%
    \usepackage{parskip}
  }{% else
    \setlength{\parindent}{0pt}
    \setlength{\parskip}{6pt plus 2pt minus 1pt}}
}{% if KOMA class
  \KOMAoptions{parskip=half}}
\makeatother
\usepackage{xcolor}
\setlength{\emergencystretch}{3em} % prevent overfull lines
\setcounter{secnumdepth}{-\maxdimen} % remove section numbering
% Make \paragraph and \subparagraph free-standing
\ifx\paragraph\undefined\else
  \let\oldparagraph\paragraph
  \renewcommand{\paragraph}[1]{\oldparagraph{#1}\mbox{}}
\fi
\ifx\subparagraph\undefined\else
  \let\oldsubparagraph\subparagraph
  \renewcommand{\subparagraph}[1]{\oldsubparagraph{#1}\mbox{}}
\fi


\providecommand{\tightlist}{%
  \setlength{\itemsep}{0pt}\setlength{\parskip}{0pt}}\usepackage{longtable,booktabs,array}
\usepackage{calc} % for calculating minipage widths
% Correct order of tables after \paragraph or \subparagraph
\usepackage{etoolbox}
\makeatletter
\patchcmd\longtable{\par}{\if@noskipsec\mbox{}\fi\par}{}{}
\makeatother
% Allow footnotes in longtable head/foot
\IfFileExists{footnotehyper.sty}{\usepackage{footnotehyper}}{\usepackage{footnote}}
\makesavenoteenv{longtable}
\usepackage{graphicx}
\makeatletter
\def\maxwidth{\ifdim\Gin@nat@width>\linewidth\linewidth\else\Gin@nat@width\fi}
\def\maxheight{\ifdim\Gin@nat@height>\textheight\textheight\else\Gin@nat@height\fi}
\makeatother
% Scale images if necessary, so that they will not overflow the page
% margins by default, and it is still possible to overwrite the defaults
% using explicit options in \includegraphics[width, height, ...]{}
\setkeys{Gin}{width=\maxwidth,height=\maxheight,keepaspectratio}
% Set default figure placement to htbp
\makeatletter
\def\fps@figure{htbp}
\makeatother

\KOMAoption{captions}{tableheading}
\makeatletter
\makeatother
\makeatletter
\makeatother
\makeatletter
\@ifpackageloaded{caption}{}{\usepackage{caption}}
\AtBeginDocument{%
\ifdefined\contentsname
  \renewcommand*\contentsname{Table of contents}
\else
  \newcommand\contentsname{Table of contents}
\fi
\ifdefined\listfigurename
  \renewcommand*\listfigurename{List of Figures}
\else
  \newcommand\listfigurename{List of Figures}
\fi
\ifdefined\listtablename
  \renewcommand*\listtablename{List of Tables}
\else
  \newcommand\listtablename{List of Tables}
\fi
\ifdefined\figurename
  \renewcommand*\figurename{Figure}
\else
  \newcommand\figurename{Figure}
\fi
\ifdefined\tablename
  \renewcommand*\tablename{Table}
\else
  \newcommand\tablename{Table}
\fi
}
\@ifpackageloaded{float}{}{\usepackage{float}}
\floatstyle{ruled}
\@ifundefined{c@chapter}{\newfloat{codelisting}{h}{lop}}{\newfloat{codelisting}{h}{lop}[chapter]}
\floatname{codelisting}{Listing}
\newcommand*\listoflistings{\listof{codelisting}{List of Listings}}
\makeatother
\makeatletter
\@ifpackageloaded{caption}{}{\usepackage{caption}}
\@ifpackageloaded{subcaption}{}{\usepackage{subcaption}}
\makeatother
\makeatletter
\@ifpackageloaded{tcolorbox}{}{\usepackage[skins,breakable]{tcolorbox}}
\makeatother
\makeatletter
\@ifundefined{shadecolor}{\definecolor{shadecolor}{rgb}{.97, .97, .97}}
\makeatother
\makeatletter
\makeatother
\makeatletter
\makeatother
\ifLuaTeX
  \usepackage{selnolig}  % disable illegal ligatures
\fi
\IfFileExists{bookmark.sty}{\usepackage{bookmark}}{\usepackage{hyperref}}
\IfFileExists{xurl.sty}{\usepackage{xurl}}{} % add URL line breaks if available
\urlstyle{same} % disable monospaced font for URLs
\hypersetup{
  pdftitle={Protokoll sykkeltest},
  colorlinks=true,
  linkcolor={blue},
  filecolor={Maroon},
  citecolor={Blue},
  urlcolor={Blue},
  pdfcreator={LaTeX via pandoc}}

\title{Protokoll sykkeltest}
\author{}
\date{}

\begin{document}
\maketitle
\ifdefined\Shaded\renewenvironment{Shaded}{\begin{tcolorbox}[interior hidden, borderline west={3pt}{0pt}{shadecolor}, breakable, sharp corners, boxrule=0pt, frame hidden, enhanced]}{\end{tcolorbox}}\fi

Deltakere: 7 mannlige deltaker ble rekruttert til prosjektet(
Gjennomsnitts alder 25,71 år. Kroppshøgde 181,28 og med e kroppsvekt 75,
72 kg). Dette var alle personer som trener regelmessig, men deres
erfaring med testing på sykkel og trening på sykkel variert innad i
gruppen.

Testing Prosjektets testdager besto av 4 dager, der halvparten av gruppa
ble testa vær dag. Test dag 1 og 2 var test 1(t1) og test dag 3 og 4 var
(t2). Med dette sikra vi at alle testpersonene fikk en dag mellom test 1
og test 2. På hviledagen fikk alle personene beskjed om at de kunne
trene va de ville av rolig trening. Dette for å sikre at de ikke fikk
utvikling på hviledagen. For alle deltakerne ble det prøvd å gjøre
testdag 1 og 2 helt identiske.

For vær deltaker startet testdagen med en 7 min lang oppvarming på
ergometer sykkel, med en gradvis økning i opplevd anstrengelse(Borg).
Objektene syklet 3 min på 11, 2 min på 13 og 2 min på 15/16 i Borg.

Del 1: Styrke Styrketesten som var enn knebøy power test. Denne besto av
et oppvarmingssett, der personen gjennomførte tre løft med bare stang.
For å måle kraftutviklingen ble det brukt en Muselabb(?) Selve testen
ble gjennomført ved en ytre belastning som var 30 \%, 60 \% og 75 \% av
egen kroppsvekt. Deltakerne fikk tre forsøk per belastning, der det
beste forsøket per belastning ble stående som tellende.

Del 2 sykkel Etter styrketest ble forsøkspersone tatt rett inn til
sykkeltest. Her gjennomførte de en tredelt test, som besto av 2
submaksimale drag, Vo2max test og en MAOD-test.

Submaksimale drag: Forsøkspersonene gjennomførte to submaksimale drag på
4 min. De submaksimale dragene ble sykla på 100 W og på 150 W. For en av
forsøkspersonene ble det gjort tilpassninger på W motstand for å få en
mer opptimal test. Han fikk 75 w på den første belastninger og 125 på
den andre belastningen. Dette ble gjort på grunn at forsøkspersonen
hadde liten erfaring ved testing på sykkel. For vært drag starta en med
å sykle 1,5 min uten slange i munnen.

TESTTEST

Generelt om sykkeltest:

\begin{itemize}
\tightlist
\item
  Forsøkspersonen sykler på Lode Excaibur
\item
  Tilnærmet likt tidspunkt på døgnet (+/- 2 timer)
\item
  Mest mulig lik tilbakemelding og engasjement hver gang

  \begin{itemize}
  \tightlist
  \item
    Lite tilbakemeldinger under de submaksimale dragene, og mye
    engasjement under VO2maks-test (spesielt mot slutten)
  \end{itemize}
\item
  Ingen opplysninger om VO2 underveis, men de får vite wattbelastning,
  samt se tiden og tråkkfrekvens underveis i makstesten
\end{itemize}

Forberedelser:

\begin{itemize}
\tightlist
\item
  Kalibrering:

  \begin{itemize}
  \tightlist
  \item
    Ambient Conditions, luftfuktighet og temperatur

    \begin{itemize}
    \tightlist
    \item
      Velg ``Ambient conditions''
    \item
      Sjekk luftfuktighet og temperatur på gradestokken
    \item
      Trykk ``F1'' for å endre luftfuktighet og temperatur, og trykk
      ``F12'' for å lagre
    \end{itemize}
  \item
    Volum calibration

    \begin{itemize}
    \tightlist
    \item
      Sett i ``trippel V'' i miksekammer
    \item
      Sett i ``sample line'' med teipbit øverst, teipbit pekende skrått
      oppover
    \item
      Finn fram en slange og fest den ene enden til det åpne hullet på
      framsiden av maskinen
    \item
      Fest den andre enden av slangen til volum-kalibreringspumpen
    \item
      Velg ``Volum Calibration''
    \item
      Trykk ``F1'' for å starte kalibrering
    \item
      Dra rolig fram og tilbake spaken på pumpen, forsøk å følge grafen
      som kommer opp på skjermen slik at rytmen blir jevn. Samtidig er
      det viktig at man trekker helt inn og helt ut slik at hele volumet
      pumpes i miksekammeret
    \item
      Pump helt til det kommer opp tall i høyre marg på skjermen
    \item
      Se på verdiene for O2- og CO2. Kalibreringen er godkjent ved en
      feilmargin på 1.0 \% (alt mellom 99.0 og 101.0 er godkjent)
    \item
      Dersom det ikke er godkjent, trykk ``F9'', og kalibrer på nytt
    \item
      Dersom godkjent, trykk ``F12''
    \end{itemize}
  \item
    Gas calibration

    \begin{itemize}
    \tightlist
    \item
      Velg ``Gas calibration''
    \item
      Åpne gassflaska
    \item
      Trykk ``F1'' og la kalibreringen gå helt til det kommer opp tall i
      høyre marg på skjermen
    \item
      Se på verdiene for O2- og CO2. Kalibreringen er godkjent ved en
      feilmargin på 1.0 (alt mellom -1.0 og 1.0 er godkjent)
    \item
      Skru på gassflaska igjen
    \item
      Dersom ikke godkjent, trykk ``F9'' og kalibrer på nytt
    \item
      Dersom godkjent, trykk ``F12'' for å lagre
    \end{itemize}
  \item
    Sett sammen munnstykket og finn fram neseklype
  \item
    Gjør klart slange og teip til å feste slangen til sykkelen
  \end{itemize}
\item
  Ta vekta til personen (uten sykkelsko), og trekk fra 0,3 kg

  \begin{itemize}
  \tightlist
  \item
    Legg inn personen på data

    \begin{itemize}
    \tightlist
    \item
      Trykk ``New Patient''
    \item
      Før inn etternavn, fornavn, id (initialer og fødselsdato uten
      punktum), fødselsdato, kjønn, høyde og vekt (tatt før
      sykkeltesten)
    \end{itemize}
  \item
    Still inn sykkelen til forsøkspersonen

    \begin{itemize}
    \tightlist
    \item
      Bytt til riktig pedaltype
    \item
      Still inn krankarm (172,5 cm)
    \item
      Still inn setehøyde og -lengde, styrelengde og -lengde

      \begin{itemize}
      \tightlist
      \item
        Lagre sittestilling
      \end{itemize}
    \end{itemize}
  \item
    Fest den ene enden av slangen til maskinen og den andre til
    munnstykket, tape fast slangen til sykkelen
  \item
    Ta på tape på forsøkspersonen sin nese
  \item
    Ta på pulsbelte på forsøkspersonen (dersom den har) og start økt på
    pulsklokke, og forsikre om at pulsen er koblet til klokka
  \item
    Gjør klart VO2-opptak
  \item
    Trykk på ``Mixing Chamber''
  \item
    Kontroller at det står ``small mouthpiece'' og ``30 sek delta time''
    i vinduet som kommer opp, trykk ``OK''
  \item
    Trykk ``F1'' for å klargjøre opptak
  \item
    Trykk ``F1'' for å starte test, og start klokke +1 sek etter start
  \end{itemize}
\end{itemize}

Nå er alt klargjort til å starte sykkeltestene. Testen starter med to
submakismale drag:

\begin{itemize}
\tightlist
\item
  To drag på 4 minutter
\item
  Drag 1: 80 watt (jenter) og 100 watt (gutter)
\item
  Etter 1,5 minutt ta i munnstykket og ta på neseklype. Det skal være på
  plass før det har gått 2 minutter
\item
  Forsøkspersonen sykler med neseklype og munnstykket i de siste 2
  minuttene (skal ha neseklype og munnstykket i før det har gått 2 min)
\item
  Noter ned tråkkfrekvens og puls hvert 30.sekund i plotteskjemaet (regn
  ut gjennomsnitt)
\item
  Nullstill klokka
\item
  Spør om opplevd anstrengelse på Borgs-skala og noter ned
\item
  Drag 2: 120 W (jenter) og 150 W (gutter)
\item
  Gjennomfør andre belastningstrinn på samme måte som beskrevet for det
  første
\end{itemize}

Pause:

\begin{itemize}
\tightlist
\item
  2 min pause sittende i ro på sykkelen
\end{itemize}

VO2maks-test:

\begin{itemize}
\tightlist
\item
  Gi beskjed om at t forsøkspersonen skal sykle til utmattelse (rpm
  \textless{} 60). «Du skal ha så mange målinger som mulig, men hvert
  sekund gir bedre prestasjon»
\item
  Starter på 160 W (jenter) og 200 W (gutter)
\item
  Økning med 20 W (jenter) og 25 W (gutter) hvert minutt til utmattelse
  (rpm \textless{} 60)
\item
  Fri tråkkfrekvens (rpm)
\item
  Skal ha neseklype og munnstykket igjennom hele testen
\item
  Får måling hvert 30. sekund, så oppmuntre forsøkspersonen til å jobbe
  for å nå flest mulig
\item
  Noter ned alt av målinger i skjema
\item
  Nullstill klokka når personen har avsluttet testen
\item
  Spør om anstrengelse på Borg-skala og noter ned rett etter test
\item
  Snittet av de to høyeste målingene defineres som VO2maks
\end{itemize}

Pause:

\begin{itemize}
\tightlist
\item
  Personen får 5 minutter mellom VO2maks-test til start på MAOD-test
\item
  Det første minuttet etter avsluttet VO 2maks-test sitter personen helt
  i ro
\item
  4 min pause på 50 W
\item
  Valgfri tråkkfrekvens, men skal ha lik ved neste test
\end{itemize}

MAOD-test:

\begin{itemize}
\tightlist
\item
  Sykkelen settes på fritt program på PC
\item
  VO2maks-testen brukes til å sette startwatt: Dersom person
  gjennomførte 30 sek eller mer starter personen på denne
  trappetrinnsbelastningen. Under dette starter de på siste fullførte
  belastningen. Den belastningen som brukes ved første test, er uansett
  lik ved neste test. Selv om man sykler lenger eller kortere på
  VO2maks-testen
\item
  Personen starter med slangen i munnen
\item
  Starter med flying start, fra 50 W. Belastningen settes klar på
  maskinen, og er klar når testleder gir beskjed at testen er klar
\item
  Sykle så lenge som mulig på denne belastningen
\item
  Testen er over når man ikke klarer å holde mer enn 60 RPM
\item
  Spør om anstrengelse på Borgs-skala rett etter avsluttet test
\item
  Noterer ned

  \begin{itemize}
  \tightlist
  \item
    Hvor lenge man syklet i sekunder
  \item
    Gjennomsnitt VO2
  \item
    Effekt (W)
  \end{itemize}
\end{itemize}

Tiltak for å sikre god reliabilitet:

\begin{itemize}
\tightlist
\item
  Tråkkfrekvensen ved første submaksimale belastningstrinn skal gjentas
  ved MAOD-testen og ved både de submaksimale trinnene og MAOD-testen på
  test 2
\item
  Samme testleder for hver enkelt forsøksperson ved begge tester
\item
  Godkjent kalibrering av volum og luft settes til +/- 1,0
\item
  Lik belastning og lengde på pause før MAOD-test begge dager
\item
  Testene gjennomføres på omtrent samme tidspunkt for hver forsøksperson
\item
  Hele testen gjennomføres sittende
\end{itemize}

Forberedelser til deltakerne:

\begin{itemize}
\tightlist
\item
  Ingen hard trening dagen før test
\item
  Bare rolig trening mellom hver test
\item
  Siste måltid (og eventuelle mellommåltid) før test skal være likt, og
  til samme tidspunkt (+/- 2 timer)
\end{itemize}

Behandling av data:

\begin{itemize}
\item
  Etter gjennomført tester må vi samle inn dataen vi skal bruke for å
  gjøre statistiske analyser
\item
  Vi noterer ned VO2 etter de submaksimale dragene
\item
  Vi regner ut VO2.rel.max og VO2.max under VO2maks-testen
\item
  Samtidig noterer vi ned andre verdifulle målinger etter test, slik som
  hr.max, W.max, rer.max, bf.max, V'E.max hvor lenge personen syklet, og
  hvilken watt personen avsluttet på og opplevd anstrengelse (Borg)
\item
  Etter MAOD-testen regner vi ut VO2.max, oksygenkravet ved belastning
  under MAOD-test (L/min), det totale okysgenkravet som må dekkes (L),
  akkumulert oksygenopptak på testene (L), akkumulert oksygengjeld og
  prosent av arbeidet som dekkes anaerobt (\%)
\item
  Samtidig noterer vi ned hvor lenge personen syklet (i sek) og opplevd
  anstrengelse (Borg) og hr.max

  o,
\end{itemize}



\end{document}
